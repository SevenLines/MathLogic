\documentclass[a4paper,10pt]{report}

\usepackage[utf8]{inputenc}
\usepackage[T2C]{fontenc}
\usepackage[russian]{babel}
\usepackage{fontenc}
\usepackage{graphicx}
\usepackage{multicol}
\usepackage{mathtext}
\usepackage{verbatim}
\usepackage[usenames,dvipsnames]{color}

\definecolor{light-gray}{gray}{0.95}

\newcommand{\hlight}[1]{\colorbox{light-gray}{#1}}

\righthyphenmin=2
\oddsidemargin=0pt
\textwidth=140mm

\date{\today}
\author{Каташевцев}
\title{Математическая логика}

\newcounter{task}[chapter]
\newcommand{\z}{\par\addtocounter{task}{1}%
 \noindent\textbf{Задача \arabic{chapter}.\arabic{task}. }}

\begin{document}

\begin{titlepage}

\begin{center}
 \large Иркутский Государственный Университет
 \\[4.5cm]
 \huge Сборник задач для практических занятий 
 \\[0.6cm]
 \large по курсу <<Математическая Логика>>
 \\[3.6cm]
\end{center}
 

\begin{flushright}
\begin{minipage}{0.5\textwidth}
 \begin{flushright}
  \emph{Автор:} Каташевцев М. Д.\\
 \end{flushright}
 
 \end{minipage}
\end{flushright}
\begin{center}
 \vfill 
 
 {\large \LaTeX}\\[0.4cm]
 {\large \the\year г.}
 \thispagestyle{empty}
\end{center}

\end{titlepage}

\tableofcontents

\chapter{Формулы исчисления высказываний}

\begin{multicols}{2}

\z Доказать что следующие формы эквивалентны:
\begin{enumerate}
 \item $A \vee B$ и $\neg(\bar{A} \wedge \bar{B})$
 \item $A \wedge B$ и $\neg(\bar{A} \vee \bar{B})$
 \item $(A \wedge B) \vee C $ и $ (A \vee C) \wedge (B \vee C) $
 \item $(A \vee B) \wedge C $ и $ (A \wedge C) \vee (B \wedge C) $
\end{enumerate}

 
  \z Расставить скобки и построить таблицу истинности для форм:
 \begin{enumerate}
  \item $\neg A \to A \wedge B$
  \item $ A \vee \neg B \to C \equiv A $
  \item $A \vee B \wedge C \to D$
 \end{enumerate}

\z Определить, является ли каждая из следующих форм тавтологией, противоречием 
или
ни тем и ни другим:
 \begin{enumerate}
  \item  $A\equiv(A \vee A)$
  \item $ (A\to B) \to ((B \to C) \to (A \to C))$
  \item $((A \to B) \wedge B) \to A $
  
  \item $A \wedge (\neg (A \vee B ))$
  \item $(A \to B) \equiv ( \neg A \vee B)$
  \item $(A \to B) \equiv \neg (A \wedge \neg B)$
 \end{enumerate}
 
\z Выразить через
 \begin{enumerate}
  \item $\vee, \neg$ связки $\wedge, \to$
  \item $\wedge, \neg$ связки $\vee, \to$
  \item $\to, \neg$ связки $\wedge, \vee$
  \item $\downarrow$ связки $\wedge, \to, \neg$
  \item $|$ связки $\wedge, \to, \neg$
 \end{enumerate}
 \z Построить КНФ, ДНФ, СКНФ и СДНФ:
 \begin{enumerate}
  \item $ X \to ( Y \to Z )$
  \item $\neg(X \vee Z) \wedge (X \to Y)$
  \item $( X \to  Y) \to Z $
  \item $(X \equiv Y) \wedge \neg (Z \to T)$
 \end{enumerate}



\end{multicols}
 
\chapter{Теория множеств. Отношения}
\section{Теория множеств}

\begin{multicols}{2}

 \z Доказать эквивалентность:
 \begin{enumerate}
  \item $\emptyset \cap X$ и $\emptyset$
  \item $ (X \cup Y) \cap Z$ и $ (X \cap Z) \cup (Y \cap Z) $
  \item $ \neg(X \cup Y)$ и $\bar{X} \cap \bar{Y} $
  \item $X \cap ( Y \cup \bar{Y})$ и $X$
  \item $X \cap ( Y \setminus X)$ и $\emptyset$
  \item  $ X \setminus Y $ и $X \setminus (X \cap Y) $
  \item $ (X \setminus Y) \setminus Z $ и $ (X \setminus Z) \setminus (Y 
\setminus Z)$
 \item $(X \cup Y) \setminus Z$ и $(X \setminus Z) \cup (Y \setminus Z)$
 \item $A \cap  (B \setminus C)$ и $ (A \cap B) \setminus (A \cap C)$ и $(A 
\cap B) \setminus C$
 \end{enumerate}

\section{Общие понятия об отношениях}
\z Построить декартово произведение:
 \begin{enumerate}
  \item $X \times X $, где $X = \{1, 2, 3, 4 \}$
  \item $X \times Y $, где $X = \{1, 2, 3, 4 \}$, а $Y = \{5,6,7\}$
  \item $X \times Y $, где $X = \{b, a, c \}$, а $Y = \{x,y,z\}$
 \end{enumerate}
 
\z Построить бинарные отношения <<$>$>>, <<$<$>> и <<$=$>> заданные на:
\begin{enumerate}
 \item на множестве $X = \{5, 6, 7, 8 \}$
 \item на декартовом произведении $X \times Y $, где $X = \{1, 2, 3, 4 \}$, а 
$Y = \{5,6,7\}$
 \item на декартовом произведении $X \times Y $, где $X = \{b, a, c \}$, а $Y = 
\{c,y,z\}$
\end{enumerate}

\z Построить тернарное отношение $\beta$ заданное на множества $S = 
\{1,2,3,4\}$, истинное для $x,y,z \in S$ тогда и только тогда когда $x<y<z$.

\z Построить унарное отношение $\alpha$ (свойство) заданное на множестве $S~=~
\{ \textrm{А ... Я} \}$ истинное для $x \in S$ тогда и только тогда когда $x$ 
-- гласная.

\z Построить тернарное отношение $w$ заданное на множестве $S~=~
\{ \textrm{А ... Я} \}$ истинное для $a,b,c \in S$ тогда и только тогда когда 
$abc$ -- некоторое слово из трех букв

\begin{comment}
\z Пусть $\alpha$, $\beta$ -- отношения. Доказать истинность следующих 
утверждений;
\begin{enumerate}
 \item $a(\alpha \cup \beta)b \Leftrightarrow a\alpha b \vee a\beta b$
 \item $a(\alpha \wedge \beta)b \Leftrightarrow a\alpha b \wedge a\beta b$
 \item $a \alpha' b \Leftrightarrow \neg a \alpha b$
\end{enumerate}
\end{comment}

\z Построить следующие отношения
\begin{enumerate}
 \item $> \cup =$
 \item $(> \cup <) \setminus =$
 \item $\geq \cap  \leq$
\end{enumerate}

\z Построить произведение отношений заданных на множестве $X=\{1,2,3,4\}$:
\begin{enumerate}
 \item $<$ и $=$
 \item $<$ и $<$
 \item $>$ и $<$
 \item $<$ и $>$
 
 \item $>$ и $>$
 \item $=$ и $<$
\end{enumerate}

\z Доказать следующие утверждения:
\begin{enumerate}
 \item $(\alpha \cup \beta)^{-1} = \alpha^{-1} \cup \beta^{-1}$
 \item $(\alpha \cap \beta)^{-1} = \alpha^{-1} \cap \beta^{-1}$
 \item $(\alpha\beta)^{-1} = \beta^{-1}\alpha^{-1}$
 \item $(\alpha^{-1})' = (a')^{-1}$

 \item $\alpha(\beta \gamma) = (\alpha\beta) \gamma$

 \item $\alpha(\beta \cup \gamma) = \alpha\beta \cup \alpha\gamma$
 \item $(\beta \cup \gamma)\alpha = \beta\alpha \cup \gamma\alpha$
\end{enumerate}

\z Построить отношение $<^{100}$ на множестве $A = \{1,2, .. 103\}$

\end{multicols}

\section{Отношения эквивалентности}

\begin{multicols}{2}

\z Доказать что следующие отношения являются отношениями эквивалентности:
\begin{enumerate}
 \item Отношение равенства по модулю
 \item Отношения сравнимости по модулю $n$
\end{enumerate}

\z Показать что следующее отношения являются отношениями 
эквивалентности и построить матрицы инцидентности
\begin{enumerate}
 \item Отношение равенства по модулю на множестве $\{-2, -1, 0, 1, 2\}$
 \item Отношение равенства тангенсов двух углов на множестве 
 $\{ 0, \frac{\pi}{2}, \pi,\frac{3\pi}{2}\}$
 \item Отношения сравнимости по модулю 3 на множестве $\{1,10, 14, 23, 24 \}$
\end{enumerate}

\z Пусть $\alpha$ и $\beta$ эквивалентности доказать следующие утверждения:
\begin{enumerate}
 \item $\alpha \cap \beta$ -- эквивалентность
 \item $\alpha \beta$ -- эквивалентность $\leftrightarrow \alpha$ и $\beta$ перестановочны  
\end{enumerate}

\z Построить фактор множество множества $A$ по отношению $\alpha$
\begin{enumerate}
 \item $A = \{1,2,3,4\}$, $\alpha \sim =$
 \item $A = \{1,2,3,4,5\}$, $\alpha \sim mod2$
 \item $A = \{4,7,23,56,31,45\}$, $\alpha \sim mod3$
 \item $A = \{0, \frac{\pi}{2},\frac{3\pi}{2},\pi,\frac{5\pi}{2}\}$, $a\alpha b \leftrightarrow sin(a) = sin(b)$ 
\end{enumerate}

\end{multicols}

\chapter{Формулы исчисления предикатов}

\begin{multicols}{2}

\z Выразить через логические операции
\begin{enumerate}
 \item $\forall xP(x)$
 \item $\exists xP(x)$
\end{enumerate}

\z Докажите эквивалентность
\begin{enumerate}
 \item $\neg \exists xP(x)$ и $\forall x\neg P(x)$
 \item $\neg \forall xP(x)$ и $\exists  x\neg P(x)$
\end{enumerate}

\z Используя формулы исчисления предикатов построить следующие высказывания
\begin{enumerate}
 \item Все люди умеют летать
 \item Любой житель Европы свободно владеет английским, арабским или китайским (2-мя способами) 
 \item Все планеты солнечной системы вращаются вокруг солнца (планеты, космические объекты)
 \item Некоторые люди не умеют летать
 \item У каждой планеты есть своя звезда вокруг которой она кружится  (планеты, космические объекты)
 \item Некоторые люди в силу определенных обстоятельств не любят летать
 \item Только на планетах с атмосферой можно обнаружить воду
\end{enumerate}

\z Расшифровать следующие высказывания
\begin{enumerate}
 \item $\forall a_1 \forall a_2 (\forall b(b \in a_1 \leftrightarrow b \in a_2) \to a_1=a_2)$
 \item $\exists a \forall b(b \notin a)$
 \item $\exists a: (\emptyset \in a \wedge \forall b(b \in a \leftrightarrow b \cup \{b\} \in a)$
 \item $\forall a_1 \forall a_2 \exists c\forall b(b \in a \leftrightarrow (b=a_1 \vee b=a_2))$
\end{enumerate}

\z Привести к предваренное нормальной форме, если $A$ не содержит свободных
вхождений переменной $x$
\begin{enumerate}
 \item $A \wedge \forall xB(x)$
 \item $A \vee \forall xB(x)$ 
 \item $A \wedge \exists xB(x)$
 \item $A \vee \exists xB(x)$ 
 \item $\forall xB(x) \wedge A$
 \item $\forall xB(x) \vee A $  
 \item $\exists xB(x) \wedge A$
 \item $A \to \exists xB(x)$
 \item $A \to \forall xB(x)$   
 \item $\exists xB(x) \to A$  
 \item $\forall xB(x) \to A$
\end{enumerate}

\end{multicols}

\pagebreak
\z Привести к предваренное нормальной форме
\begin{enumerate}
 \item $\forall xP(x) \to P(y)$
 \item $\forall xP(x) \to P(x,y)$
 \item $\forall xP(x) \to \exists y \exists x P(x,y)$
 \item $\forall x(P(x) \to Q(x)) \to (\exists xP(x) \to \exists yQ(y)$
 \item $\forall x(P(x) \to \forall xQ(x)$
 \item $\forall xQ(x,y) \vee (\exists x Q(x,x) \to \forall z(R(t,z) \to \exists xQ(x,x) )$
 \item $\forall yQ(y,z) \to \exists xR(x,t,z)$
 \item $\forall yQ(x,y) \to R(x,x)$
 \item $(P(y) \wedge Q(x)) \to \neg \forall yR(y,z)$
 \item $\forall x (A(x) \to  \forall y(A(x,y)\to \neg\forall zA(y,z)))$
 \item $A(x,y) \to \exists y[A(y) \to (\exists xA(x) \to A(y))]$
\end{enumerate}



\z Используя формулы исчисления предикатов построить следующие высказывания
\begin{enumerate}
 \item Существует ровно один элемент x такой что P(x)
 \item Существует не более одного элемента x такого что P(x) 
 \item Существует не более двух элементов x таких что P(x) 
 \item Между любыми двумя различными точками на прямой лежит по меньшей мере одна, с ними не совпадающая
 \item Через две различные точки на плоскости проходит единственная прямая
 \item Существование четных число
 \item Существование нечетных число 
 \item Существование простых числа
 \item Существование периодических функций
\end{enumerate}

\begin{multicols}{1}

\z Докажите что следующие формулы являются тавтологиями логики предикатов
\begin{enumerate}
 \item $\forall xP(x) \to P(y)$
 \item $P(y) \to \exists x P(x)$
 \item $\forall xP(x) \to \exists xP(x)$
 \item $\exists yP(y) \to P(x)$
 \item $\forall x\forall yP(x,y) \to \forall xP(x,x)$
 \item $\exists xP(x,x) \to \exists x\exists y P(x,y)$
 \item $\forall x\forall yQ(x,y) \leftrightarrow \forall y\forall xQ(x,y) $
 \item $\exists x\exists yQ(x,y) \leftrightarrow \exists y\exists xQ(x,y) $
 \item $\forall x\exists z(F(x,y) \vee \neg F(z,y))$
 \item $\forall x\exists y\forall z((P(x) \wedge P(y)) \to Q(z))$ 
 \item $\exists y\forall xP(x,y) \to \forall\exists P(x,y)$
\end{enumerate}

\end{multicols}

\chapter{Вывод в ФИВ}

\begin{multicols}{2}

\z Построить вывод 
\begin{enumerate}
 \item $\vdash A \to (B \to A)$
 \item $\vdash C \to (D \to C)$
 \item $\vdash B \to ((A \to B) \to B)$ 
 \item $\vdash B \to (A \to (B \to A))$
 \item \hlight{$\vdash B \to (B \to (A \to B))$}
 \item $\vdash (A \to B) \to (A \to A)$
 \item \hlight{$\vdash (A \to A) \to (A \to A)$}
 \item $\vdash (\neg A \to A) \to A$
 \item $\vdash A \to (B \to ((C \to A) \to B))$
 \item $\vdash A \to A$
 \item $\vdash (A \to A) \to (B \to (A \to A))$ 
 \item $\vdash F \to ((H \to F) \to (G \to G))$
 \item \hlight{$\vdash F \to ((G \to G) \to (H \to F))$}
\end{enumerate}

\z Построить вывод используя гипотезы:
\begin{enumerate}
 \item $A, A \to B \vdash B$
 \item $A, B \vdash C \to A$
 \item $A \vdash B \to A$
 \item $B \vdash C \to (A \to B)$
 \item $B, B \to C \vdash D \to C$
 \item $A \to (A \to C) \vdash A \to C$
 \item $A \to (B \to C), B \vdash A \to C$
 \item \hlight{$A \to (B \to C), A \to B, A \vdash C$}
 \item $\neg B \to A, \neg A \vdash B$
 \item \hlight{$\neg B \to \neg A, A \vdash B$}
 \item $A \to B, B \to C \vdash A \to C$
 \item $A \to (B \to C), B \vdash A \to C$
 \item $\neg B \to A \vdash (\neg B \to \neg A) \to B$
 \item $\vdash A \to (B \to (C \to A))$
 \item $F, G, F \to (G \to H) \vdash H$
 \item \hlight{$A, \neg A \vdash B$}

\end{enumerate}

\z Построить вывод используя теорему дедукции:
\begin{enumerate}
 \item $F \to ((F \to G) \to G)$
 \item $(F \to G) \to ((H \to F) \to (H \to G)$
 \item $(F \to (F \to G)) \to (F \to G)$
 \item $(F \to G) \to ((F \to (G \to H)) \to (F \to H)$
 \item $\neg \neg L \to L$
 \item $U \to \neg \neg U$ 
 \item $(\neg B \to \neg A) \vdash (A \to B)$
 \item $(\neg B \to A) \vdash (\neg A \to B)$
 \item $(B \to \neg A) \vdash (A \to \neg B)$
 \item $(B \to A) \vdash (\neg A \to \neg B)$
 \item $ A \to B, \neg A \to B \vdash B$  
 \item $ E \to D, E \to \neg D \vdash E$   
\end{enumerate}

\end{multicols}

\chapter{Теория алгоритмов}
\section{Машина Тьюринга}

\begin{multicols}{2}

\z Построить машины тьюринга:
\begin{enumerate}

\item Добавление единицы справа
\item \hlight{Удаление единицы слева}
\item Заменяющая все единицы на нули
\item Заменяющая все нули на пустые символы
\item Сложение единиц
\item Увеличение целого числа на единицу
\item Вычитание единицы из целого числа
\item Подсчет единиц (дешифратор)
\item Выбор наибольшой последовательности из единиц
\item \hlight{Выбор наименьшей последовательности из единиц}
\item Вычитание единиц
\item \hlight{Умножение единиц}
\item Десятичное число в единицы (шифратор)
\item Удвоение единиц
\item Перестановка последовательностей из единиц

\end{enumerate}

\end{multicols}

\section{Конечный Автомат}

\begin{multicols}{2}
\z Построить конечный автомат
\begin{enumerate}
\item Распознающий строку не содержащую единиц
\item Распознающий числа заданные в двоичной системе счисления делящиеся на 4
\item Распознающий числа заданные в двоичной системе счисления делящиеся на 16
\item Распознающий числа заданные в единичной системе счисления делящиеся на 2
\item Распознающие числа с плавающей запятой
\item Распознающий предложения содержащие четное кол-во слов
\item Распознающий адреса e-mail 
\item Распознающий ip-адреса
\end{enumerate}


\z Построить МНР машину
\begin{enumerate}
\item Перестановка регистров
\item Сложение положительных чисел
\item Остаток от деления
\item Целочисленное деления
\item Умножение чисел
\item Вычитание положительных чисел
\end{enumerate}

\end{multicols}

\end{document}


