\documentclass[10pt,a4paper]{article}
\usepackage[utf8]{inputenc}
\usepackage{amsmath}
\usepackage{amsfonts}
\usepackage{amssymb}
\usepackage{graphicx}
\usepackage[russian]{babel}

\setlength{\parindent}{0pt}
\begin{document}

\section*{Схема базы данных}

\section*{Проект базы данных}

\textbf{Bands} (Музыкальные группы)
\begin{enumerate}
\item id -- уникальный идентификатор 
\item title -- название группы
\item genre\_id -- жанр в котором играет группа (ссылка на таблицу жанров)
\end{enumerate}
Таблица \textbf{Bands} образует отношение, которое является подмножеством декартова произведения
$$I_{Bands} \times T \times I_{Genres}$$
где $T$ есть множество всевозможных названий групп, $I_{Bands}, I_{Genres} \subseteq Z$

\

\textbf{Genres} (Музыкальные жанры)
\begin{enumerate}
\item id -- уникальный идентификатор жанра
\item name -- название жанра
\item year -- год основания жанра
\end{enumerate}
Таблица \textbf{Genres} образует отношение, которое является подмножеством декартова произведения
$$I_{Genres} \times N \times Y$$
где $N$ есть множество всевозможных названий жанров, $I_{Genres},Y \subseteq Z$ 

\section*{Содержимое таблиц}
\begin{tabular}{c|c|c}
id & title & genre\_id \\ \hline
1	& Beatles, The	& 1 \\ 
2	& Nirvana	& 4 \\ 
3	& Bjork	& 2 \\ 
4	& Elvis Presley	& 1 \\ 
5	& Alice in Chains	& 4 \\
6	& Lana del Rey	& 2 \\ 
7	& Little Richard	& 1 \\ 
8	& Babes In Toyland	& 4 \\ 
9	& Elisane	& 2 \\ 
\end{tabular}


\begin{tabular}{c|c|c}
id & name & year \\ \hline
1	& Rock-n-roll	& 1950\\
2	& Trip hop	& 1990 \\
4	& Grunge	& 1980 \\

\end{tabular}

\section*{Запросы}

\end{document}