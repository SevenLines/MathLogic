\documentclass[10pt,a4paper]{article}
\usepackage[utf8]{inputenc}
\usepackage{amsmath}
\usepackage{amsfonts}
\usepackage{amssymb}
\usepackage{graphicx}
\usepackage{multicol}
\usepackage{times}
\usepackage[russian]{babel}


\usepackage{geometry}
\geometry{
	left=6mm,
	right=6mm,
	top=6mm,
	bottom=6mm,			
}

\setlength{\columnsep}{12mm}

\begin{document} 
\begin{multicols}{2}
\pagenumbering{gobble}
\noindent
Вариант 1
\begin{enumerate}
\item Доказать/опровергнуть выполнимость формул, не строя таблиц истинности
$$(P\vee(\neg P\vee(N\vee(N\to P)))\&\neg(N\to P)$$
\item Доказать/опровергнуть что следующие формулы являются тавтологиями, не строя таблиц истинности
$$((L\to I)\&(F\to E))\to((L\& F)\to(I\&E))$$
\item Доказать/опровергнуть логическое следование
\begin{equation*}\begin{split}C\to(A\to L),\;& (L\& U)\to D,\; \\& M\to(U\&\neg D)\;\models\; C\to (A\to M)\end{split}\end{equation*}
\item Найти КНФ(или ДНФ) и достроить до СКНФ(или СДНФ), в дополнение найти СКНФ(СДНФ) с помощью таблицы истинности и сравнить с полученной с помощью достройки
$$(S\& U)\equiv(\neg N \vee S)$$
\end{enumerate}
Вариант 2
\begin{enumerate}
\item Доказать/опровергнуть выполнимость формул, не строя таблиц истинности
$$\neg((((M\to K)\to K)\to K)\to K)\&\neg(M)$$
\item Доказать/опровергнуть что следующие формулы являются тавтологиями, не строя таблиц истинности
$$((F\to I)\&(\neg R\to E)\&\neg(I\vee E))\to\neg(F\& R)$$
\item Доказать/опровергнуть логическое следование
\begin{equation*}\begin{split}(R\& A)\to M,\;& M\to(A\&\neg M),\; \\& D\to (E\to M)\;\models\; D\to(E\to R)\end{split}\end{equation*}
\item Найти КНФ(или ДНФ) и достроить до СКНФ(или СДНФ), в дополнение найти СКНФ(СДНФ) с помощью таблицы истинности и сравнить с полученной с помощью достройки
$$((N\to S)\to(S\& U))\wedge((S\to U)\to(N\to S))$$
\end{enumerate}
Вариант 3
\begin{enumerate}
\item Доказать/опровергнуть выполнимость формул, не строя таблиц истинности
$$\neg(A\to B)\&(B\vee(\neg B\vee(A\vee(A\to B)))$$
\item Доказать/опровергнуть что следующие формулы являются тавтологиями, не строя таблиц истинности
$$((C\to A)\&(V\to E))\to((C\& V)\to(A\&E))$$
\item Доказать/опровергнуть логическое следование
\begin{equation*}\begin{split}M\to(S\&\neg H),\;& F\to (A\to M),\; \\& F\to(A\to L)\;\models\; (L\& S)\to H\end{split}\end{equation*}
\item Найти КНФ(или ДНФ) и достроить до СКНФ(или СДНФ), в дополнение найти СКНФ(СДНФ) с помощью таблицы истинности и сравнить с полученной с помощью достройки
$$(C\& A)\equiv(\neg T \vee C)$$
\end{enumerate}

\vfill\columnbreak

Вариант 4
\begin{enumerate}
\item Доказать/опровергнуть выполнимость формул, не строя таблиц истинности
$$\neg(M)\&\neg((((M\to Y)\to Y)\to Y)\to Y)$$
\item Доказать/опровергнуть что следующие формулы являются тавтологиями, не строя таблиц истинности
$$((S\to U)\&(\neg R\to F)\&\neg(U\vee F))\to\neg(S\& R)$$
\item Доказать/опровергнуть логическое следование
\begin{equation*}\begin{split}P\to (A\to M),\;& P\to(A\to L),\; \\& (L\& N)\to T\;\models\; M\to(N\&\neg T)\end{split}\end{equation*}
\item Найти КНФ(или ДНФ) и достроить до СКНФ(или СДНФ), в дополнение найти СКНФ(СДНФ) с помощью таблицы истинности и сравнить с полученной с помощью достройки
$$((T\to C)\to(C\& A))\wedge((C\to A)\to(T\to C))$$
\end{enumerate}
Вариант 5
\begin{enumerate}
\item Доказать/опровергнуть выполнимость формул, не строя таблиц истинности
$$(E\vee(\neg E\vee(H\vee(H\to E)))\&\neg(H\to E)$$
\item Доказать/опровергнуть что следующие формулы являются тавтологиями, не строя таблиц истинности
$$((G\to I)\&(F\to T))\to((G\& F)\to(I\&T))$$
\item Доказать/опровергнуть логическое следование
\begin{equation*}\begin{split}B\to(U\to R),\;& (R\& S)\to H,\; \\& M\to(S\&\neg H)\;\models\; B\to (U\to M)\end{split}\end{equation*}
\item Найти КНФ(или ДНФ) и достроить до СКНФ(или СДНФ), в дополнение найти СКНФ(СДНФ) с помощью таблицы истинности и сравнить с полученной с помощью достройки
$$(Y\& E)\equiv(\neg S \vee Y)$$
\end{enumerate}
Вариант 6
\begin{enumerate}
\item Доказать/опровергнуть выполнимость формул, не строя таблиц истинности
$$\neg((((H\to E)\to E)\to E)\to E)\&\neg(H)$$
\item Доказать/опровергнуть что следующие формулы являются тавтологиями, не строя таблиц истинности
$$((G\to I)\&(\neg R\to L)\&\neg(I\vee L))\to\neg(G\& R)$$
\item Доказать/опровергнуть логическое следование
\begin{equation*}\begin{split}(I\& H)\to T,\;& M\to(H\&\neg T),\; \\& L\to (G\to M)\;\models\; L\to(G\to I)\end{split}\end{equation*}
\item Найти КНФ(или ДНФ) и достроить до СКНФ(или СДНФ), в дополнение найти СКНФ(СДНФ) с помощью таблицы истинности и сравнить с полученной с помощью достройки
$$((S\to Y)\to(Y\& E))\wedge((Y\to E)\to(S\to Y))$$
\end{enumerate}

\end{multicols}

\pagebreak


\begin{multicols}{2}



Вариант 1
\begin{enumerate}
\item Постройте вывод (воспользуйтесь 2 и 1 аксиомами): $$O,\;O\to(N\to E) \vdash O\to E$$
\item Отношения, определение, свойства, связь с таблицами БД
\item Можно ли назвать аксиоматическую теорию L естественным (формальным) языком. Обоснуйте.
\end{enumerate}
\noindent\rule{\columnwidth}{0.1pt} \\
    


Вариант 2
\begin{enumerate}
\item Постройте вывод (воспользуйтесь 3 и 1 аксиомами): $$C,\; \neg H \to \neg C \vdash H$$
\item Множества (пустое множество, подмножество, операции над множествами). Связь с ФИВ.
\item Сформулируйте связь между формальным понятием отношения и тем отношением которые вы испытываете к находящимся в этой аудитории
\end{enumerate}
\noindent\rule{\columnwidth}{0.1pt} \\
    


Вариант 3
\begin{enumerate}
\item Постройте вывод (воспользуйтесь 2 и 1 аксиомами): $$N\to(Y\to A),\;N \vdash N\to A$$
\item Аксиоматическая теория L Гильберта (определение, аксиомы)
\item С помощью пустого множества постройте множество содержащее бесконечное количество элементов (напомню, что в множестве не может быть пары одинаковых элементов)
\end{enumerate}
\noindent\rule{\columnwidth}{0.1pt} \\
    


Вариант 4
\begin{enumerate}
\item Постройте вывод (воспользуйтесь 3 и 1 аксиомами): $$\neg N \to \neg Y,\; Y \vdash N$$
\item Вывод в формулах исчисления высказываний (ФИВ). Примеры (если удалось доказать пример выше то можно не писать).
\item Сформулируйте связь между формальным понятием отношения и тем отношением которые вы испытываете к находящимся в этой аудитории
\end{enumerate}
\noindent\rule{\columnwidth}{0.1pt} \\
    
\columnbreak\vfill

Вариант 5
\begin{enumerate}
\item Постройте вывод (воспользуйтесь 2 и 1 аксиомами): $$D,\;D\to(U\to M) \vdash D\to M$$
\item Теорема дедукции в ФИВ
\item С помощью пустого множества постройте множество содержащее бесконечное количество элементов (напомню, что в множестве не может быть пары одинаковых элементов)
\end{enumerate}
\noindent\rule{\columnwidth}{0.1pt} \\
    


Вариант 6
\begin{enumerate}
\item Постройте вывод (воспользуйтесь 3 и 1 аксиомами): $$L,\; \neg A \to \neg L \vdash A$$
\item Отношения, определение, свойства, связь с таблицами БД
\item Можно ли назвать аксиоматическую теорию L естественным (формальным) языком. Обоснуйте.
\end{enumerate}
\noindent\rule{\columnwidth}{0.1pt} \\
    


Вариант 7
\begin{enumerate}
\item Постройте вывод (воспользуйтесь 2 и 1 аксиомами): $$S\to(U\to N),\;S \vdash S\to N$$
\item Множества (пустое множество, подмножество, операции над множествами). Связь с ФИВ.
\item Сформулируйте связь между формальным понятием отношения и тем отношением которые вы испытываете к находящимся в этой аудитории
\end{enumerate}
\noindent\rule{\columnwidth}{0.1pt} \\
    


Вариант 8
\begin{enumerate}
\item Постройте вывод (воспользуйтесь 3 и 1 аксиомами): $$\neg R \to \neg U,\; U \vdash R$$
\item Аксиоматическая теория L Гильберта (определение, аксиомы)
\item С помощью пустого множества постройте множество содержащее бесконечное количество элементов (напомню, что в множестве не может быть пары одинаковых элементов)
\end{enumerate}
\noindent\rule{\columnwidth}{0.1pt} \\


\end{multicols}

\pagebreak

\begin{multicols}{2}
Вариант 1
\begin{enumerate}
\item Доказать:
$$B\,\dot{-}\,U=B'$$
\item Построить отношение $(\alpha\cdot\beta\cup\beta\cdot\alpha)\setminus\gamma$:
$$\begin{array}{l} \alpha=\{ (n, s), (b, o), (b, y), (w, s) \} \\ \beta=\{ (s, e), (s, i), (s, d), (m, b), (a, b) \} \\ \gamma=\{ (n, e), (w, i) \} \end{array}$$
\item Привести к предваренной нормальной форме: \:\\
$$\exists lR(l,o) \to \neg \forall o( E(o,l) \wedge \exists l\exists v D(v, l))$$
\item Проанализируйте рассуждение::\\
Все бегуны -- спортсмены. Ни один спортсмен не курит. Следовательно, ни один курящий не является бегуном
\item Построить вывод:\\
$$K \to L \vdash \neg K \to \neg L$$
\end{enumerate}
\noindent\rule{\columnwidth}{0.1pt} \\

Вариант 2
\begin{enumerate}
\item Доказать:
$$B\setminus A = B\,\dot{-}\,(B\cap A)$$
\item Построить отношение $(\alpha\cdot\beta\cup\beta\cdot\alpha)\setminus\gamma$:
$$\begin{array}{l} \alpha=\{ (u, s), (t, h), (d, p), (k, h) \} \\ \beta=\{ (o, d), (h, i), (h, a), (h, y), (c, u) \} \\ \gamma=\{ (t, y), (k, a) \} \end{array}$$
\item Привести к предваренной нормальной форме: \:\\
$$\forall mW(m,a) \to \neg \forall a( E(a,m) \wedge \forall m\forall d B(d, m))$$
\item Проанализируйте рассуждение::\\
Некоторые змеи ядовиты. Ужи -- змеи. Следовательно, ужи -- ядовиты. 
\item Построить вывод:\\
$$K, \neg K \vdash \neg L$$
\end{enumerate}

\noindent\rule{\columnwidth}{0.1pt} \\
\columnbreak\vfill

Вариант 3
\begin{enumerate}
\item Доказать:
$$A\,\dot{-}\,\emptyset=A$$
\item Построить отношение $(\alpha\cdot\beta\cup\beta\cdot\alpha)\setminus\gamma$:
$$\begin{array}{l} \alpha=\{ (c, y), (e, c), (v, l), (o, c) \} \\ \beta=\{ (c, n), (c, t), (i, c), (c, m), (g, v) \} \\ \gamma=\{ (e, t), (o, n) \} \end{array}$$
\item Привести к предваренной нормальной форме: \:\\
$$\exists fL(f,a) \to \neg \exists a( S(a,f) \wedge \forall f\exists r D(r, f))$$
\item Проанализируйте рассуждение::\\
Все студенты ИГУ -- жители Иркутской области. Некоторые жители Иркутской области -- пенсионеры. Следовательно, некоторые студенты ИГУ -- пенсионеры
\item Построить вывод:\\
$$M \to \neg T \vdash T \to \neg M$$
\end{enumerate}
\noindent\rule{\columnwidth}{0.1pt} \\

Вариант 4
\begin{enumerate}
\item Доказать:
$$A\setminus B = A\,\dot{-}\,(A\cap B)$$
\item Построить отношение $(\alpha\cdot\beta\cup\beta\cdot\alpha)\setminus\gamma$:
$$\begin{array}{l} \alpha=\{ (k, e), (r, o), (n, y), (b, y) \} \\ \beta=\{ (y, g), (y, u), (y, t), (d, r), (l, k) \} \\ \gamma=\{ (n, t), (b, u) \} \end{array}$$
\item Привести к предваренной нормальной форме: \:\\
$$\forall gC(g,i) \to \neg \exists i( A(i,g) \wedge \exists g\forall t T(t, g))$$
\item Проанализируйте рассуждение::\\
Все сильные шахматисты знают теорию шахматной игры.Иванов -- так себе шахматист. Следовательно он не знает теорию шахматной игры.
\item Построить вывод:\\
$$A \to (B \to C) \vdash B \to (A \to C)$$
\end{enumerate}
\noindent\rule{\columnwidth}{0.1pt} \\

\end{multicols}

\end{document}